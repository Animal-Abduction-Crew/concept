\subsection{Anforderungen}

Fortfolgend werden die verbleibenden nicht funktionalen Anforderungen an den \ac{AAR} in einer Tabelle dokumentiert, wobei die aus \autoref{sec:use_cases} bekannte Skala wieder zur Priorisierung herangezogen wird.

\begin{dhbwtable}{%
    caption	= Nicht funktionale Anforderungen,
    label	= tab:non_func_reqs,
    float = h!
}
    \begin{tabularx}{\textwidth}{llp{0.55\textwidth}}
        \toprule
        \textbf{Anforderung}    & \textbf{Priorität} & \textbf{Beschreibung} \\\midrule
        Hardware                & 5 & Raspberry Pi 3 Model B, Motore, H-Brücken, Pi Cam, Ultraschallsensoren, Infrarotsensoren, Lautsprecher, Mikrofon, Taster müssen zuverlässig funktionieren. \\
        Betriebsystem           & 5 & Raspbian Buster Lite \\
        Programmiersprache      & 5 & Python 3.7 \\
        Softwarebibliotheken    & 5 & pigpio, OpenCV (inkl. OpenCV-contrib), rhasspy \\
        Durchlaufzeit           & 4 & Nach 60 Sekunden muss das gesuchte Tier außerhalb des Spielfelds sein. Alternativ nach spätestens 90 Sekunden. \\
        \ac{CPU}-Temperatur     & 4 & Die \ac{CPU}-Temperatur darf während einer Objekterkennung nicht 70 Grad Celsius überschreiten.\\
        Spracherkennungszeit    & 4 & Die Verarbeitung eines Sprachbefehls darf nicht länger als zwei Sekunden dauern.\\\bottomrule
    \end{tabularx}    
\end{dhbwtable}