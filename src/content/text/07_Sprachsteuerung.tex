\subsection{Exkurs: Sprachsteuerung}

Für die Sprachsteuerung des \acp{AAR} wird die Open-Source Software \enquote{Rhasspy} genutzt.
Durch die Integration der ebenfalls quelloffenen freien Bibliotheken \enquote{porcupine} für die Wakeword-Erkennung, \enquote{Pocketsphinx} für Umwandlung von Sprache in Text und \enquote{fuzzywuzzy} für die Erkennung der Absicht eines Sprachbefehls, gelingt es Rhasspy eine Lösung bereitzustellen, welche offline, ohne Einbindung von Diensten Dritter funktioniert.\ifootcite{rhasspy_about}

Rhasspy bietet mehrere Schnittstellen, welche genutzt werden können um erkannte Absichten in Sprachbefehlen ein andere Dienste weiterzuleiten.
Für die Integration Rhasspys in den \ac{AAR} bieten sich grundlegend zwei Schnittstellen an:
Zum einen kann ein Shellskript aufgerufen werden, wobei die Eigenschaften der erkannten Absicht über Standardeingabe-Parameter übergeben werden können.
Alternativ kann Rhasspy auch so konfiguriert werden, dass erkannte Absichten als \ac{JSON} Objekte über eine \ac{HTTP} POST-Request an einen Webserver weitergereicht werden können.\ifootcite{rhasspy_intent_handling}

Für die Interaktion mit dem \ac{AAR} sind vorerst die Befehle \enquote{Hole das Tier X!} und \enquote{Führe eine Selbstcheck durch!} angedacht, wobei beide Absichten in mehreren Formulierungen in deutscher Sprache eingerichtet werden.
Die Rückmeldung bei erfolgreicher Erkennung eines Befehl erfolgt per Lautsprecher.
So berichtet der \ac{AAR} bspw. nach dem Befehl \enquote{Hole das Tier X!} welches Tier erkannt wurde.
Für die Umwandlung von Text zu Sprache wird die Bibliothek \enquote{pyttsx3} eingesetzt, welche ebenfalls vollständig offline funktionsfähig ist.
Zusätzlich kann der Lautsprecher während der Entwicklung der Software des \acp{AAR} dazu genutzt werden, den Ablauf der implementierten Algorithmen während Tests besser nachvollziehen zu können, indem der Roboter über den Lautsprecher ausgibt, welche Anweisungen er gerade ausführt.
Dazu muss lediglich ein existentes Logging-System um einen \enquote{Log2Speech} Adapter erweitert oder ein neues minimales Logging-System implementiert werden.
