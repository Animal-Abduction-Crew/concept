\subsection{Use-Cases}

\begin{dhbwtable}{%
    caption	= Use-Case: Tier auswählen,
    label	= tab:use_case_select_animal,
    source	= Eigene Darstellung,
    float = h!
}
    \begin{tabularx}{\textwidth}{lX}
        \toprule
        \textbf{Merkmal}     & \textbf{Beschreibung}  \\\midrule
        Auslöser     & Roboter befindet sich in der Startsituation.\\
        Komponenten  & Mikrofon, Lautsprecher, Raspberry\\
        Beschreibung & Anwender spricht das Wakeword aus, gefolgt von einer Tierauswahl. Der \ac{AAR} erkennt die Auswahl und startet die Tiersuche.\\\bottomrule
    \end{tabularx}    
\end{dhbwtable}

\begin{dhbwtable}{%
    caption	= Use-Case: Tier auffinden,
    label	= tab:use_case_find_animal,
    source	= Eigene Darstellung,
    float = h!
}
    \begin{tabularx}{\textwidth}{lX}
        \toprule
        \textbf{Merkmal}     & \textbf{Beschreibung}  \\\midrule
        Auslöser     & Tier wurde per Spracheingabe ausgewählt.\\
        Komponenten  & Kamera, Antrieb links, Antrieb rechts, Ultraschallsensor links, Ultraschallsensor rechts\\
        Beschreibung & Der \ac{AAR} fährt geradeaus durch das Spielfeld und versucht mit der Kamera das gesuchte Tier zu identifizieren und bleibt stehen, wenn das richtige Tier gefunden wurde.\\\bottomrule
    \end{tabularx}    
\end{dhbwtable}

\begin{dhbwtable}{%
    caption	= Use-Case: Auf Tier ausrichten,
    label	= tab:use_case_turn_to_animal,
    source	= Eigene Darstellung,
    float = h!
}
    \begin{tabularx}{\textwidth}{lX}
        \toprule
        \textbf{Merkmal}     & \textbf{Beschreibung}  \\\midrule
        Auslöser     & Das gesuchte Tier wurde gefunden.\\
        Komponenten  & Kamera, Antrieb links, Antrieb rechts\\
        Beschreibung & Der \ac{AAR} dreht sich so lange, bis er direkt in Fahrtrichtung auf das gesuchte Tier ausgerichtet ist.\\\bottomrule
    \end{tabularx}    
\end{dhbwtable}

\begin{dhbwtable}{%
    caption	= Use-Case: Zum Tier fahren,
    label	= tab:use_case_go_to_animal,
    source	= Eigene Darstellung,
    float = h!
}
    \begin{tabularx}{\textwidth}{lX}
        \toprule
        \textbf{Merkmal}     & \textbf{Beschreibung}  \\\midrule
        Auslöser     & Der \ac{AAR} ist auf das gesuchte Tier ausgerichtet.\\
        Komponenten  & Kamera, Antrieb links, Antrieb rechts, Infrarotsensor\\
        Beschreibung & Der \ac{AAR} fährt auf das erkannt Tier zu, bis der Infrarotsensor feststellt, dass das Tier direkt vor den Greifern des \acp{AAR} steht.\\\bottomrule
    \end{tabularx}    
\end{dhbwtable}

\begin{dhbwtable}{%
    caption	= Use-Case: Tier greifen,
    label	= tab:use_case_grab_animal,
    source	= Eigene Darstellung,
    float = h!
}
    \begin{tabularx}{\textwidth}{lX}
        \toprule
        \textbf{Merkmal}     & \textbf{Beschreibung}  \\\midrule
        Auslöser     & Das gesuchte Tier befindet sich direkt vor den Greifern des \acp{AAR}. \\
        Komponenten  & Greiferantrieb links, Greiferantrieb rechts, Schließtaster \\
        Beschreibung & Der \ac{AAR} schließt den linken und rechten Greifer, bis der Schließtaster ausgelöst wird.\\\bottomrule
    \end{tabularx}    
\end{dhbwtable}

\begin{dhbwtable}{%
    caption	= Use-Case: Tier aus dem Spielfeld fahren,
    label	= tab:use_case_move_animal,
    source	= Eigene Darstellung,
    float = h!
}
    \begin{tabularx}{\textwidth}{lX}
        \toprule
        \textbf{Merkmal}     & \textbf{Beschreibung}  \\\midrule
        Auslöser     & Das gesuchte Tier befindet sich in den geschlossenen Greifern des \acp{AAR}.\\
        Komponenten  & Antrieb links, Antrieb rechts, Liniensensor links, Liniensensor rechts\\
        Beschreibung & \\\bottomrule
    \end{tabularx}    
\end{dhbwtable}

\begin{dhbwtable}{%
    caption	= Use-Case: Tier loslassen,
    label	= tab:use_case_release_animal,
    source	= Eigene Darstellung,
    float = h!
}
    \begin{tabularx}{\textwidth}{lX}
        \toprule
        \textbf{Merkmal}     & \textbf{Beschreibung}  \\\midrule
        Auslöser     &\\
        Komponenten  & \\
        Beschreibung & \\\bottomrule
    \end{tabularx}    
\end{dhbwtable}

\begin{dhbwtable}{%
    caption	= Use-Case: Vom Tier entfernen,
    label	= tab:use_case_leave_animal,
    source	= Eigene Darstellung,
    float = h!
}
    \begin{tabularx}{\textwidth}{lX}
        \toprule
        \textbf{Merkmal}     & \textbf{Beschreibung}  \\\midrule
        Auslöser     &\\
        Komponenten  & \\
        Beschreibung & \\\bottomrule
    \end{tabularx}    
\end{dhbwtable}