\subsection{Use-Cases}\label{sec:use_cases}

Nachfolgend werden die funktionalen Anforderungen an den \ac{AAR} in Form von Use-Cases beschrieben.
Die Priorisierung erfolgt anhand einer Skala von 1 (niedrig) bis 5 (hoch).

\begin{dhbwtable}{%
    caption	= Use-Case: Tier auswählen,
    label	= tab:use_case_select_animal,
    float   = h!
}
    \begin{tabularx}{\textwidth}{lX}
        \toprule
        \textbf{Merkmal}     & \textbf{Beschreibung}  \\\midrule
        Auslöser     & Roboter befindet sich in der Startsituation.\\
        Komponenten  & Mikrofon, Lautsprecher, Raspberry Pi\\
        Beschreibung & Anwender spricht das Wakeword aus, gefolgt von einer Tierauswahl. Der \ac{AAR} erkennt die Auswahl und startet die Tiersuche.\\
        Priorität    & 5 \\\bottomrule
    \end{tabularx}    
\end{dhbwtable}

\begin{dhbwtable}{%
    caption	= Use-Case: Tier auffinden,
    label	= tab:use_case_find_animal,
    float   = h!
}
    \begin{tabularx}{\textwidth}{lX}
        \toprule
        \textbf{Merkmal}     & \textbf{Beschreibung}  \\\midrule
        Auslöser     & Tier wurde per Spracheingabe ausgewählt.\\
        Komponenten  & Kamera, Antrieb links, Antrieb rechts, Ultraschallsensor links, Ultraschallsensor rechts\\
        Beschreibung & Der \ac{AAR} fährt geradeaus durch das Spielfeld und versucht mit der Kamera das gesuchte Tier zu identifizieren und bleibt stehen, wenn das richtige Tier gefunden wurde.\\
        Priorität    & 5 \\\bottomrule
    \end{tabularx}    
\end{dhbwtable}

\begin{dhbwtable}{%
    caption	= Use-Case: Auf Tier ausrichten,
    label	= tab:use_case_turn_to_animal,
    float   = h!
}
    \begin{tabularx}{\textwidth}{lX}
        \toprule
        \textbf{Merkmal}     & \textbf{Beschreibung}  \\\midrule
        Auslöser     & Das gesuchte Tier wurde gefunden.\\
        Komponenten  & Kamera, Antrieb links, Antrieb rechts\\
        Beschreibung & Der \ac{AAR} dreht sich so lange, bis er direkt in Fahrtrichtung auf das gesuchte Tier ausgerichtet ist.\\
        Priorität    & 5 \\\bottomrule
    \end{tabularx}    
\end{dhbwtable}

\begin{dhbwtable}{%
    caption	= Use-Case: Zum Tier fahren,
    label	= tab:use_case_go_to_animal,
    float   = h!
}
    \begin{tabularx}{\textwidth}{lX}
        \toprule
        \textbf{Merkmal}     & \textbf{Beschreibung}  \\\midrule
        Auslöser     & Der \ac{AAR} ist auf das gesuchte Tier ausgerichtet.\\
        Komponenten  & Kamera, Antrieb links, Antrieb rechts, Infrarotsensor\\
        Beschreibung & Der \ac{AAR} fährt auf das erkannt Tier zu, bis der Infrarotsensor feststellt, dass das Tier direkt vor den Greifern des \acp{AAR} steht.\\
        Priorität    & 5 \\\bottomrule
    \end{tabularx}    
\end{dhbwtable}

\begin{dhbwtable}{%
    caption	= Use-Case: Tier greifen,
    label	= tab:use_case_grab_animal,
    float   = h!
}
    \begin{tabularx}{\textwidth}{lX}
        \toprule
        \textbf{Merkmal}     & \textbf{Beschreibung}  \\\midrule
        Auslöser     & Das gesuchte Tier befindet sich direkt vor den Greifern des \acp{AAR}. \\
        Komponenten  & Greiferantrieb links, Greiferantrieb rechts, Schließtaster \\
        Beschreibung & Der \ac{AAR} schließt den linken und rechten Greifer, bis der Schließtaster ausgelöst wird.\\
        Priorität    & 4 \\\bottomrule
    \end{tabularx}    
\end{dhbwtable}

\begin{dhbwtable}{%
    caption	= Use-Case: Tier aus dem Spielfeld fahren,
    label	= tab:use_case_move_animal,
    float   = h!
}
    \begin{tabularx}{\textwidth}{lX}
        \toprule
        \textbf{Merkmal}     & \textbf{Beschreibung}  \\\midrule
        Auslöser     & Das gesuchte Tier befindet sich in den geschlossenen Greifern des \acp{AAR}.\\
        Komponenten  & Antrieb links, Antrieb rechts, Liniensensor links, Liniensensor rechts\\
        Beschreibung & Der \ac{AAR} fährt geradeaus bis ein Liniensensor die Linie detektiert. Auf der Seite, auf der die Linie erkannt wurde, wird der Antrieb abgeschalten, bis der andere Liniensensor die Linie erkennt. Danach fährt der \ac{AAR} geradeaus, bis er sich mindestens 10cm von der Linie entfernt hat.\\
        Priorität    & 5 \\\bottomrule
    \end{tabularx}    
\end{dhbwtable}

\begin{dhbwtable}{%
    caption	= Use-Case: Tier loslassen,
    label	= tab:use_case_release_animal,
    float   = h!
}
    \begin{tabularx}{\textwidth}{lX}
        \toprule
        \textbf{Merkmal}     & \textbf{Beschreibung}  \\\midrule
        Auslöser     & Der \ac{AAR} befindet sich mit dem gesuchten Tier mindestens 10cm außerhalb des Spielfelds.\\
        Komponenten  & Greiferantrieb links, Greiferantrieb rechts, Öffnungstaster\\
        Beschreibung & Der \ac{AAR} schließt den linken und rechten Greifer, bis der Öffnungstaster ausgelöst wird.\\
        Priorität    & 4 \\\bottomrule
    \end{tabularx}    
\end{dhbwtable}

\begin{dhbwtable}{%
    caption	= Use-Case: Vom Tier entfernen,
    label	= tab:use_case_leave_animal,
    float   = h!
}
    \begin{tabularx}{\textwidth}{lX}
        \toprule
        \textbf{Merkmal}     & \textbf{Beschreibung}  \\\midrule
        Auslöser     & Der \ac{AAR} steht mit geöffnetem Greifarmen vor dem außerhalb des Spielfelds stehenden gesuchten Tieres.\\
        Komponenten  & Antrieb links, Antrieb rechts\\
        Beschreibung & Der \ac{AAR} entfernt sich 10cm rückwärts vom abgestellten Tier.\\
        Priorität    & 3 \\\bottomrule
    \end{tabularx}    
\end{dhbwtable}