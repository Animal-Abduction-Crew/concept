\subsection{Aufbau des Roboters}

Der Name \ac{AAR} (zu deutsch \enquote{Tier-Entführungsroboter}) leitet sich von der zuvor beschriebenen Aufgabe ab, Tiere zu finden und zu bewegen.

Folgend wird nun beschrieben, wie der \ac{AAR} aufgebaut ist und welche Sensoren bzw. Komponenten für welche Zwecke eingesetzt werden.
Außerdem wird auf die Belegung der einzelnen \ac{GPIO} Pins des Raspberry Pis eingegangen.

\subsubsection{Skizze}

\dhbwFigure{%
    caption	= Animal Abduction Robot,
    label	= fig:aar,
    path	= content/assets/aar.jpg,
    source	= Eigene Darstellung
}

\autoref{fig:aar} zeigt den \ac{AAR} von oben.
Auffallend ist der größe Greifmechanismus, welcher die vordere Hälfte des \ac{AAR} einnimmt.
Die hintere Hälfte des \ac{AAR} ist schmal gehalten und beinhaltet sämtliche Elektronik, wie den Akku, den Raspberry Pi, ein Breadboard und mehrere Sensoren. 

Ursrünglich war vorgesehen Ultraschallsensoren zur Ergänzung der Objekterkennung einzusetzen.
Falls sich ein Tier seitlich des \ac{AAR} befände, könnte es mit einem Ultraschallsensor erkannt werden, welcher eine höhere Entfernung detektierten würde, als die Entfernung zu den gekippten Tischen, welche das Spielfeld eingrenzen, da ein Plüschtier die Ultraschallsignale des Sensors absorbieren würde.
Da die Objekterkennung aber zuverlässig funktioniert sind aktuell keine Ultraschallsensoren auf dem \ac{AAR} installiert.

\begin{dhbwtable}{%
    caption	= Aufbau Roboter,
    label	= tab:show_parts_at_aar,
    source	= Eigene Darstellung,
    float   = h!
}
    \begin{tabularx}{\textwidth}{lXX}
        \toprule
        \textbf{Nummer}     & \textbf{Bauteil}              & \textbf{Beschreibung} \\\midrule
        1   	            & Akku                          & Stromversorgung für Motoren und Raspberry Pi \\
        2                   & Raspberry Pi                  & Recheneinheit und Ansteuerungsmodul \\
        3                   & Ultraschallsensor rechts      & Erkennung falls sich ein Tier Rechts des AAR befindet \\
        4                   & Ultraschallsensor links       & Erkennung falls sich ein Tier Links des AAR befinde \\
        5                   & H-Brücke für den Greifer      & Steuert die Motoren für den Greifer an \\
        6                   & H-Brücke für den Antrieb      & Steuert die Motoren des Antriebes an \\
        7                   & Spannungswandler              & regelt die Spannung des Akkus für den Raspberry auf 5 Volt runter \\
        8                   & Motor+Getriebe Greifer rechts & bewegt den rechten Teil des Greifarms \\
        9                   & Motor+Getriebe Greifer links  & bewegt den linken Teil des Greifarms \\
        10                  & Infrarot links                & erkennt eine Linie auf der linken Seite \\
        11                  & Infrarot rechts               & erkennt eine Linie auf der rechten Seite \\
        12                  & Infrarot mitte                & erkennt ob sich ein Tier vor dem AAR befindet \\
        13                  & Kamera                        & wird für die Objekterkennung genutzt \\
        14                  & Antriebsmotor rechts          & dreht das rechte Rad \\
        15                  & Antriebsmotor links           & dreht das linke Rad \\\bottomrule
    \end{tabularx}    
\end{dhbwtable}

\autoref{tab:show_parts_at_aar} listet die Hauptkomponenten des \ac{AAR}, sowie eine Beschreibung derer Aufgaben.

\subsubsection{Schaltplan}

Für die Umsetzung der spezifizieren Funktionalitäten, wie etwa der Greifmechanismus oder die alternative Suche nach Plüschtieren mit Ultraschallsensoren, müssen einige Komponenten über die \ac{GPIO}-Schnittstelle der Raspberry Pis angesteuert werden.
In \autoref{app:wiring_diagram} ist ein Schaltplan zu sehen, welcher die Verkabelung aller Komponenten über ein Breadboard mit dem Raspberry Pi zeigt und \autoref{tab:gpio_pins} beschreibt, die Funktionalitäten der verwendeten Pins.

\begin{dhbwtable}{%
    caption	= Belegung der GPIO Pins,
    label	= tab:gpio_pins
}
    \begin{tabular}{llll}
        \toprule
        \textbf{Bauteil} & \textbf{\ac{GPIO}-Pin} & \textbf{Funktion} \\\midrule
        Motor rechts   	    & 18        & Fahrtrichtung vorwärts\\
                            & 15        & Fahrtrichtung rückwärts\\
                            & 14 (PWM)  & Geschwindigkeitsregelung\\
        Motor links   	    & 7         & Fahrtrichtung vorwärts\\
                            & 1         & Fahrtrichtung rückwärts\\
                            & 12 (PWM)  & Geschwindigkeitsregelung\\    
        Motoren Greifer   	& 5         & Greifer öffnen\\
                            & 6         & Greifer schließen\\
                            & 13 (PWM)  & Geschwindigkeitsregelung\\
        Ultraschall rechts 	& 11        & Trigger\\
                            & 9         & Echo\\
        Ultraschall links   & 20        & Trigger\\
                            & 16        & Echo \\ 
        Infrarot links      & 2         & Input\\
        Infrarot rechts     & 3         & Input\\
        Infrarot mitte      & 4         & Input\\\bottomrule
    \end{tabular}    
\end{dhbwtable}
