\subsection{Aufbau des Roboters}

% erklärung name (aar = animal abduction robot)

\subsubsection{Skizze}

\dhbwFigure{%
    caption	= Animal Abduction Robot,
    label	= fig:aar,
    path	= content/assets/aar.jpg,
    source	= Eigene Darstellung
}

\begin{dhbwtable}{%
    caption	= Aubau Roboter,
    label	= tab:show_parts_at_aac,
    source	= Eigene Darstellung,
}
    \begin{tabular}{llll}
        \toprule
        \textbf{Nummer}     & \textbf{Bauteil}      &\textbf{Beschreibung}   \\\midrule
        1   	            & Akku                  & Stromversorgung für Motoren und Raspberry Pi\\
        2                   & Raspberry Pi          & Recheneinheit und Ansteuerungsmodul \\
        3                   & Ultraschallsensor rechts   & Erkennung falls sich ein Tier Rechts des ACC befindet  \\
        4                   & Ultraschallsensor links   & Erkennung falls sich ein Tier Links des ACC befindet  \\
        5                   & H-Brücke für den Greifer   & Steuert die Motoren für den Greifer an   \\
        6                   & H-Brücke für den Antrieb   & Steuert die Motoren des Antriebes an   \\
        7                   & Spannungswandler   & regelt die Spannung des Akkus für den Raspberry auf 5 Volt runter  \\
        8                   & Motor+Getriebe Greifer rechts   & bewegt den rechten Teil des Greifarms  \\
        9                   & Motor+Getriebe Greifer links   & bewegt den linken Teil des Greifarms  \\
        10                  & Infrarot links   & erkennt eine Linie auf der linken Seite  \\
        11                  & Infrarot rechts   & erkennt eine Linie auf der rechten Seite  \\
        12                  & Infrarot mitte   & erkennt ob sich ein Tier vor dem AAC befindet  \\
        13                  & Kamera   & wird für die Objekterkennung genutzt  \\
        14                  & Antriebsmotor rechts   & dreht das rechte Rad  \\
        15                  & Antriebsmotor links   & dreht das linke Rad  \\

        
        \bottomrule
    \end{tabular}    
\end{dhbwtable}


% bild
% nennung und beschreibung der komponenten
% warum diese komponenten?; ultraschall-geschichte...

\subsubsection{Schaltplan}

\dhbwFigure{%
    caption	= Schaltplan Respberry Pi,
    label	= fig:SRP,
    path	= content/assets/Roboter_Steckplatine.png,
    source	= Eigene Darstellung
}

% fritzing wiring diagram