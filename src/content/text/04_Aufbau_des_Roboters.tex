\subsection{Aufbau des Roboters}

Der Name \ac{AAR} (zu deutsch \enquote{Tier-Entführungsroboter}) leitet sich von der zuvor beschriebenen Aufgabe ab, Tiere zu finden und zu bewegen.

Folgend wird nun beschrieben, wie der \ac{AAR} aufgebaut ist und welche Sensoren bzw. Komponenten für welche Zwecke eingesetzt werden.
Außerdem wird auf die Belegung der einzelnen \ac{GPIO} Pins des Raspberry Pis eingegangen.

\subsubsection{Skizze}

\dhbwFigure{%
    caption	= Animal Abduction Robot,
    label	= fig:aar,
    path	= content/assets/aar.jpg,
    source	= Eigene Darstellung
}

\autoref{fig:aar} zeigt den \ac{AAR} von oben.
Auffallend ist der größe Greifmechanismus, welcher die vordere Hälfte des \ac{AAR} einnimmt.
Die hintere Hälfte des \ac{AAR} ist schmal gehalten und beinhaltet sämtliche Elektronik, wie den Akku, den Raspberry Pi, ein Breadboard und mehrere Sensoren. 

Ursprünglich war vorgesehen Ultraschallsensoren zur Ergänzung der Objekterkennung einzusetzen.
Falls sich ein Tier seitlich des \ac{AAR} befände, könnte es mit einem Ultraschallsensor erkannt werden, welcher eine höhere Entfernung detektierten würde, als die Entfernung zu den gekippten Tischen, welche das Spielfeld eingrenzen, da ein Plüschtier die Ultraschallsignale des Sensors absorbieren würde.
Da die Objekterkennung aber zuverlässig funktioniert sind aktuell keine Ultraschallsensoren auf dem \ac{AAR} installiert.

\autoref{tab:aar_parts} listet die Hauptkomponenten des \ac{AAR}, sowie eine Beschreibung deren Aufgaben.

\begin{dhbwtable}{%
    caption	= Bauteile des \ac{AAR},
    label	= tab:aar_parts,
    source	= Eigene Darstellung,
    float   = h!
}
    \begin{tabularx}{\textwidth}{lXX}
        \toprule
        \textbf{Nummer} & \textbf{Bauteil} & \textbf{Beschreibung} \\\midrule
        1 & Akku                                & Stromversorgung der Motoren und des Raspberry Pis \\
        2 & Raspberry Pi                        & Einplatinencomputer \\
        3 & Ultraschallsensor rechts            & Plüschtiererkennung rechts \\
        4 & Ultraschallsensor links             & Plüschtiererkennung links \\
        5 & H-Brücke für den Greifer            & Steuert die Motoren der Greifer an \\
        6 & H-Brücke für den Antrieb            & Steuert die Motoren des Antriebes an \\
        7 & Spannungswandler                    & Regelt die Spannung des Akkus für den Raspberry auf 5 V \\
        8 & Motor \& Getriebe Greifer rechts    & Bewegt den rechten Greifarm \\
        9 & Motor \& Getriebe Greifer links     & Bewegt den linken Greifarm \\
        10& Infrarotsensor links                & Linke Linienerkennung \\
        11& Infrarotsensor rechts               & Rechte Linienerkennung \\
        12& Infrarotsensor mitte                & Plüschtiererkennung vorne \\
        13& Kamera                              & Objekterkennung \\
        14& Antriebsmotor rechts                & Dreht das rechte Rad \\
        15& Antriebsmotor links                 & Dreht das linke Rad \\\bottomrule
    \end{tabularx}
\end{dhbwtable}

\subsubsection{Schaltplan}

Für die Umsetzung der spezifizieren Funktionalitäten, wie etwa der Greifmechanismus oder die alternative Suche nach Plüschtieren mit Ultraschallsensoren, müssen einige Komponenten über die \ac{GPIO}-Schnittstelle der Raspberry Pis angesteuert werden.
In \autoref{app:wiring_diagram} ist ein Schaltplan zu sehen, welcher die Verkabelung aller Komponenten über ein Breadboard mit dem Raspberry Pi zeigt und \autoref{tab:gpio_pins} beschreibt, die Funktionalitäten der verwendeten Pins.

\begin{dhbwtable}{%
    caption	= Belegung der \ac{GPIO} Pins,
    label	= tab:gpio_pins
}
    \begin{tabular}{llll}
        \toprule
        \textbf{Bauteil} & \textbf{\ac{GPIO}-Pin} & \textbf{Funktion} \\\midrule
        Motor rechts   	    & 18        & Fahrtrichtung vorwärts\\
                            & 15        & Fahrtrichtung rückwärts\\
                            & 14 (PWM)  & Geschwindigkeitsregelung\\
        Motor links   	    & 7         & Fahrtrichtung vorwärts\\
                            & 1         & Fahrtrichtung rückwärts\\
                            & 12 (PWM)  & Geschwindigkeitsregelung\\    
        Motoren Greifer   	& 5         & Greifer öffnen\\
                            & 6         & Greifer schließen\\
                            & 13 (PWM)  & Geschwindigkeitsregelung\\
        Ultraschall rechts 	& 11        & Trigger\\
                            & 9         & Echo\\
        Ultraschall links   & 20        & Trigger\\
                            & 16        & Echo \\ 
        Infrarot links      & 2         & Input\\
        Infrarot rechts     & 3         & Input\\
        Infrarot mitte      & 4         & Input\\\bottomrule
    \end{tabular}    
\end{dhbwtable}
